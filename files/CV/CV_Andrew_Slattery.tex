\documentclass{clean_cv}

% Add a BibTeX-style file encoding all of your publications to include here. You can export this from Zotero. Only include
% publications you want to appear here!
\addbibresource{CV_Andrew_Slattery.bib}

\author{Andrew Slattery}
\headlineposition{}

\begin{document}
	
	\maketitle
	% In this section, you can use any of the FontAwesome icons. The commands \faCenter and \faCenterStyle have been defined to properly center the icons
	% when using the default font settings.
	%
	% You can use any of the icons listed in the fontawesome5 package documentation (https://ctan.math.utah.edu/ctan/tex-archive/fonts/fontawesome5/doc/fontawesome5.pdf)
	% If you need to specify a specific style (as is done here for the address card), you should use the two-argument \faCenterCycle command
	\begin{center}
		\begin{tabular}{lll}
			\faCenter{envelope} \href{mailto:aws46@cam.ac.uk}{aws46@cam.ac.uk}  & \faCenter{phone-alt} (+44) 07599 182341 & \faCenterStyle{regular}{address-card} Flat 51, Harvey Goodwin Gardens, Cambridge, CB4 3EZ \\
			\faCenter{orcid} \href{https://orcid.org/0009-0008-4933-821X}{0009-0008-4933-821X} & \faCenter{github} \href{https://github.com/AndrewSlattery}{AndrewSlattery} & \faCenter{globe} \url{https://aslattery.co.uk/} \\
		\end{tabular}
	\end{center}
	
	\vspace{-1.5em}
	
	\section{Employment}
	\begin{datetabular}{9em}
		\dateentry{Apr 2024 --}{
			\textbf{University of Cambridge}
			
			\textit{Research Associate}
			\begin{itemize}
				\item Supervisor: Dr Jon Sterling
				\item Funding provided by US Air Force Grant FA95502310728 `New spaces for denotational semantics'
				\item Extending the reach of denotational semantics with a homotopical and geometrical understanding of computation via higher dimensional category theory and topos theory
			\end{itemize}
		}
	\end{datetabular}
	
	\section{Education}
	
	\begin{datetabular}{9em}
		\dateentry{Oct 2020 -- Mar 2024}{
			\textbf{University of Leeds}
			
			\textit{PhD in Category Theory}
			\begin{itemize}
				\item Funding provided by EPSRC Scholarship
				\item Supervisors: Dr Nicola Gambino, Dr Andrew Brooke-Taylor
				\item Thesis Title: Commutativity of Relative Pseudomonads
			\end{itemize}
		}
		\dateentry{Oct 2019 -- Jun 2020}{
			\textbf{University of Cambridge}
			
			\textit{Masters of Mathematics (Part III) -- Pass}\\
		}
		\dateentry{Oct 2016 -- Jun 2019}{
			\textbf{University of Cambridge}
			
			\textit{Bachelor of Arts in Mathematics -- First}
		}
	\end{datetabular}
	
	\section{Research interests}
	
	My research is in higher category theory and its applications to logic and computer science. I am particularly interested in the use of categorical structures such as higher-dimensional multicategories, pseudoadjunctions and pseudoalgebras in contexts such as topos theory and fibred categories.
	
	\section{Papers}
	\nocite{*}
	\highlightauthorname{Andrew}{A.}{Slattery} 
	\begin{datetabular}{9em}
		
		\dateentry{2025}{\printbibyear{2025}} 
		\dateentry{2023}{\printbibyear{2023}} 
	\end{datetabular}
	
	\section{Talks and Posters}
	
	\begin{datetabular}{9em}
		\dateentry{2025}{
			\textbf{Logic in Computer Science (LICS) 40}, Singapore 
			
			\textit{Hofmann–Streicher lifting of fibred categories}
		}
		\dateentry{2024}{
			\textbf{Category Theory 2024}, Santiago de Compostela
			
			\textit{Bicategories of algebras for relative pseudomonads} (poster)
		}
		
		\dateentry{2023}{
			
			\textbf{University of Manchester Category Theory Seminar}, Manchester
			
			\textit{Presheaf Algebras are Cocomplete Categories}
			
			\textbf{Category Theory 2023}, Louvain-la-Neuve
			
			\textit{Pseudocommutativity and lax idempotency for relative pseudomonads} (poster)
			
			\textbf{107th Peripatetic Seminar on Sheaves and Logic (PSSL)}, Athens
			
			\textit{Pseudocommutativity for relative pseudomonads}
		}
		\dateentry{2022}{
			\textbf{Yorkshire and Midlands Category Theory Seminar 27}, Leeds
			
			\textit{Property-like structures and relative pseudomonads}
		}
		\dateentry{2021}{
			\textbf{`Proofs, Constructions, Computations and Categories' seminar}, Leeds
			
			\textit{The 2-category of algebras over a relative 2-monad}
			
			\textbf{Postgraduate Logic Seminar}, Leeds
			
			\textit{Well-quasiorders and Kruskal's Tree Theorem}
		}
	\end{datetabular}
	
	\section{Teaching and Administrative Work}
	\begin{datetabular}{9em}
		\dateentry{Oct 2024 --}{
			\textbf{University of Cambridge}
			
			\textit{Supervision tutor}
			\begin{itemize}
				\item Mathematics Part IB Analysis and Topology
				\item Mathematics Part II Graph Theory
			\end{itemize}
		}
		
		\dateentry{Oct 2021 -- Mar 2024}{
			\textbf{University of Leeds}
			
			\textit{Teaching Assistant}
			\begin{itemize}
				\item MATH1055 Numbers and Vectors
				\item MATH1050 Calculus and Mathematical Analysis
				\item MATH1026 Sets, Sequences and Series
				\item MATH1060 Introductory Linear Algebra
				\item MATH1025 Number Systems
			\end{itemize}
			
			\textit{Co-organiser}
			\begin{itemize}
				\item Yorkshire and Midlands Category Theory Seminar
				\item `Proofs, Constructions, Computations and Categories' seminar
			\end{itemize}
		}
	\end{datetabular}	
	
	\section{Scholarships and Awards}
	
	\begin{datetabular}{9em}
		\dateentry{Oct 2020 -- Apr 2024}{EPSRC PhD Scholarship}
		\dateentry{2019}{Harry Patten Prize for Best Result in Part II Mathematics at Clare}
		\dateentry{2017}{Amiya Batterji Prize for Best Result in Part I Mathematics at Clare}
	\end{datetabular}
	
\end{document}
